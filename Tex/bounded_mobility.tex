\documentclass[reprint,unsortedaddress,amsmath,amssymb,floatfix,aps,prl,showkeys]{revtex4-2}
\usepackage{graphicx}% Include figure files
\usepackage{dcolumn}% Align table columns on decimal point
\usepackage{subfigure}
\usepackage{bookmark}
% \usepackage{biblatex}
\usepackage{float}
\usepackage{url}
\usepackage{bm}% bold math
\usepackage{hyperref}% add hypertext capabilities
\usepackage[mathlines]{lineno}% Enable numbering of text and display math

\begin{document}
\title{Human mobility with a bounded memory}
\author{Gezhi Xiu, Jianying Wang, Lei Dong}
\author{Yu Liu}
\email{liuyu@urban.pku.edu.cn}
\affiliation{Institute of Remote Sensing and Geographic Information System (IRSGIS), Peking University}
\date{\today}

\begin{abstract}
    We propose a model of mobility considering exploration to new places and revisit to a limited number of frequent places.
\end{abstract}

\maketitle

Urban life is busy. People visit places with selection. Existing model has shown that a small fraction of destinations cover most of citizen's time spent. In the meantime, new sites of cites are built over time, bringing people to better choices to visit, i.e., the set of one's destinations are substitutive\cite{ref2}. On the other hand, the upper-bound tolerance of the order of urban life is limited. For example, 




% \bibliography{ref.bib}
\begin{thebibliography}{99}  
    \bibitem{ref1} Campos D, Méndez V. Recurrence time correlations in random walks with preferential relocation to visited places. Physical Review E, 2019, 99(6): 062137.
    \bibitem{ref2} Jin, C., Song, C., Bjelland, J. et al. Emergence of scaling in complex substitutive systems. Nat Hum Behav 3, 837–846 (2019) doi:10.1038/s41562-019-0638-y
\end{thebibliography}

\end{document}